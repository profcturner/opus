%
% OPUS Manual
%

% Latex 2e

\pdfcompresslevel=9
\pdfoutput=1

% Try 12 point
\documentclass[12 pt]{book}
%
% Pick one of the following font packages (if you don't like CM)
%
%\usepackage{pslatex}
\usepackage{times}
%\usepackage{palatino}
%\usepackage{newcent}
%
%
\usepackage{hyperref}
\usepackage{makeidx}
\usepackage{fancyhdr}
\usepackage[pdftex]{graphicx}
%\usepackage{amsfonts}
\usepackage{url}
\usepackage{color}
\usepackage{listings}
\usepackage{layout}
\usepackage[margin=3cm,innermargin=1.5cm,outermargin=1cm,includemp=true,marginparsep=1 cm, marginparwidth=2.5 cm, paperwidth=190 mm,paperheight=260 mm]{geometry}
\usepackage{ifthen}
%
% Define some dingbats
%
\usepackage{pifont}

%
% Change the default bullet
%
\renewcommand{\labelitemi}{%
        \raisebox{-.25ex}{\ding{43}}}

%
% Does it all have to be Arial now? <sigh>
%
\renewcommand{\familydefault}{\sfdefault}

\DeclareGraphicsExtensions{.pdf,.png,.mps}

\pdfinfo{
  /Title (OPUS Manual)
  /Author (Dr Colin Turner)
  /Producer (PDFLaTeX)
}

\makeindex

\begin{document}

% New commands, and notation
%
% Preamble for all documents
%
% Colin Turner
%

% Latex 2e

\newcommand{\UniversityName}{the University of Ulster}
\newcommand{\ShortUniversityName}{Ulster}
\newcommand{\OpusUrl}{http://opus.ulster.ac.uk}
\newcommand{\DevelopmentUrl}{http://foss.ulster.ac.uk/projects/opus/}
\newcommand{\ContactEmail}{opus@foss.ulster.ac.uk}

% Headings
%
% Headings for notes
%
% Colin Turner
%

% Latex 2e

% Headings



%
% New fancyhdr version
%
\pagestyle{fancy}
\addtolength{\headwidth}{\marginparsep}
\addtolength{\headwidth}{\marginparwidth}
\renewcommand{\chaptermark}[1]{\markboth{#1}{}}
\renewcommand{\sectionmark}[1]{\markright{\thesection\ #1}}
\fancyhf{}
\renewcommand{\headrulewidth}{0pt} % get rid of the line
\fancyhead[LE,RO]{\colorbox{cyan}{\color{white}\thepage\normalcolor}}
\fancyhead[LO]{\colorbox{cyan}{\color{white}\rightmark\normalcolor}}
\fancyhead[RE]{\colorbox{cyan}{\color{white}\leftmark\normalcolor}}
\fancypagestyle{plain}{%
\fancyhead{} % get rid of headers
\renewcommand{\headrulewidth}{0pt} % and the line
}

%
% Now some LaTeX magic from
% http://research.cmis.csiro.au/gjw/tex/docs/fancyhdr.pdf
% to make extra pages blank
%


\makeatletter
\def\cleardoublepage{\clearpage\if@twoside \ifodd\c@page\else
\hbox{}
\vspace*{\fill}
%\begin{center}
%This page intentionally contains only this sentence.
%\end{center}
\vspace{\fill}
\thispagestyle{empty}
\newpage
\if@twocolumn\hbox{}\newpage\fi\fi\fi}
\makeatother


% Listings package
\lstset{
  language=PHP,
  alsolanguage=HTML,
  basicstyle=\ttfamily,
  flexiblecolumns=true,
  lineskip=-2pt,
  commentstyle=\color{blue},
  morecomment=[s][\color{red}]{/**}{*/},
%  frame=l,
  numbers=left,
  numberstyle=\tiny,
  stepnumber=5,
  numbersep=5pt,
  backgroundcolor=\color[rgb]{0.95, 0.95, 0.95}
}


% New environments

\newcommand{\pdstext}[1]{\color{magenta}{\sffamily\bfseries #1}\normalcolor} 
\newcommand{\opustext}[1]{\color{magenta}{\sffamily\bfseries #1}\normalcolor} 
\newcommand{\pdsstudentmenu}[1]{\color{blue}{\ttfamily \bfseries #1}\normalcolor} 
\newcommand{\pdsacademicmenu}[1]{\color{red}{\ttfamily \bfseries #1}\normalcolor} 
\newcommand{\opusacademicmenu}[1]{\color{red}{\ttfamily \bfseries #1}\normalcolor} 
\newcommand{\pdsiconcd}{\marginpar{\includegraphics[scale=1.0]{png/icon_cd.png}}}
\newcommand{\pdsiconsa}{\marginpar{\includegraphics[scale=1.0]{png/icon_sa.png}}}
\newcommand{\pdsiconmc}{\marginpar{\includegraphics[scale=1.0]{png/icon_mc.png}}}
\newcommand{\pdsicons}{\marginpar{\includegraphics[scale=1.0]{png/icon_s.png}}}
\newcommand{\pdsiconpt}{\marginpar{\includegraphics[scale=1.0]{png/icon_pt.png}}}
\newcommand{\pdsbookletref}[1]
{\marginpar
  {\makebox[0 pt][l]
    {\includegraphics[scale=1.0]{png/icon_book.png}
  }
  \parbox{2 cm}{{\sffamily \bfseries \tiny #1}}}}

\newcommand{\pdstip}[1]{{\bfseries TIP!} \emph{#1}}
\newcommand{\opustip}[1]{{\bfseries TIP!} \emph{#1}}

%
% Page Dimensions, required to be custom for this job
%
\setlength{\paperheight}{260 mm}
\setlength{\paperwidth}{190 mm}
\setlength{\pdfpageheight}{\paperheight}
\setlength{\pdfpagewidth}{\paperwidth}


\title{OPUS Manual}

\author{Dr Colin Turner\\\url{c.turner@ulster.ac.uk}}

\date{October, 2006}

%%% ----------------------------------------------------------------------

\pagenumbering{roman}
\maketitle

\pagestyle{fancy}

\tableofcontents
\markboth{Contents}{Contents}

%\listoftables

%\listoffigures

%
% Introduction, unnumbered chapter
%

\setlength{\parindent}{0 cm}
\setlength{\parskip}{0.2 cm}

\chapter*{Introduction\markboth{Introduction}{}}
Welcome to the OPUS manual. As you will see this is really a work very much in progress. Please note that the most up-to-date version will always be
found at the development site for OPUS at \url{http://foss.ulster.ac.uk/projects/opus}, currently only in source repository, but hopefully we will enable
automatic builds of the documentation in the near future.

%\newpage
%\pagecolor{red}
%\pagecolor{white}
%\newpage

\chapter{How to use this workbook}
\pagenumbering{arabic}

You 
might choose to read this workbook from cover to cover,
or to dip into it as required. The following conventions
have been 
used to help you get the most of this book whatever way you choose to use it. 

\section{Signposts}
\index{Signposts}

At various points in the workbook, signposts are used in
the margins to indicate that a section is of specific
interest to a particular role you may have.

\parbox{2 cm}{\includegraphics{png/icon_cd.png}}
\parbox{9 cm}{Course Directors ({\bfseries CD}) have many varied responsibilities,
including often acting as an Adviser of Studies.}
\index{Course Director}
\medskip

\parbox{2 cm}{\includegraphics{png/icon_sa.png}}
\parbox{9 cm}{Studies Advisers ({\bfseries SA}). The PDSystem has a great deal of
dedicated support for the studies advice process.}
\index{adviser of study}
\index{studies adviser|see{adviser of study}}
\medskip

\parbox{2 cm}{\includegraphics{png/icon_mc.png}}
\parbox{9 cm}{Module Coordinators ({\bfseries MC}). Many
of the facilities useful to Course Directors are also
useful to Module Coordinators.}
\index{Module Coordinators}
\medskip

\parbox{2 cm}{\includegraphics{png/icon_s.png}}
\parbox{9 cm}{Supervisors ({\bfseries S}). For the purposes
of this workbook, supervisors are taken to include those
responsible for undergraduate projects, right up to PhD
supervision.}
\index{Supervising}
\medskip

\parbox{2 cm}{\includegraphics{png/icon_pt.png}}
\parbox{9 cm}{Placement Tutors ({\bfseries 
PT}). The PDSystem provides many possibilities for
monitoring and assessing placements, and integrates into
OPUS. OPUS is another University of Ulster on-line system
which deals comprehensively with the whole
placement process: it aids in the tasks of finding students
a placement; recording their placement information,
assessing their placement and more. More details can
be found at \url{http://opus.ulster.ac.uk}.}
\index{placement!tutors}

\section{Typographical Conventions}
\index{typographical conventions}

Text which appears on OPUS screens will be
presented \pdstext{like this}.

Where specific menu selections on OPUS are detailed they will
be shown as follows: 
\begin{itemize}
\item \pdsstudentmenu{Section -> Subsection} when the student strand is discussed; and
\item \pdsacademicmenu{Section -> Subsection} when the staff strand is being discussed.
\end{itemize}

\pdstip{Sometimes an important tip will appear like this}.

\part{Administration Guide}

\chapter{Installation}

We look at the various requirements of OPUS here and how to install it.

\section{Requirements}
\index{requirements}

OPUS can be probably be run on top of  any system that supports PHP%
\footnote{A popular web scripting language (\url{http://www.php.net})} and MySQL.%
\footnote{A relational database engine (\url{http://www.mysql.com})}

At the time of writing, specific requirements are
\begin{itemize}
  \item PHP version 5.1 or higher (since PDO is planned to be used in new code);
  \item MySQL version 4 or higher;
  \item Smarty\footnote{A templating enginer (\url{http://smarty.php.net})} version 1.3 or higher;
  \item Pear;%
    \footnote{Extensions to PHP, found at \url{http://pear.php.net}. Pear is required for DB support, and was used for mail\_mime support prior to version 3.3.0,
     but will probably be phased out in version 4.0.0 when built in PDO support is likely to be used.}
  \item Perl version 5 or higher.\footnote{A powerful scripting language, required for some offline maintainence, available from \url{http://www.perl.com}.}
\end{itemize}

\section{Operating Systems}
\index{operating systems}
\index{Debian GNU/Linux}

It should be straight forward to install OPUS on most variants of Unix, or Linux and, with perhaps some more work Microsoft Windows.

\subsection{GNU/Linux}

Various distributions of GNU/Linux, or simply Linux, as they are often called, are well suited to running OPUS. Almost all modern versions
package all or most of the requirements listed above.

The absolutely simplest way is to install OPUS on a server running Debian GNU/Linux 4.0 (Etch) or above%
\footnote{Available for many architectures from \url{http://www.debian.org}.}%
since OPUS is specifically \emph{packaged} for Debian.

Indeed, using the Debian packages will allow you to greatly circumvent many of the other issues in this, and the following chapter, since
much of the installation and packaging will be handled automatically.

To ensure automatic installation and upgrades, add the following line to your configuration for \lstinline!apt! usually found in
\lstinline!/etc/apt/sources.list!.

\begin{lstlisting}
deb http://foss.ulster.ac.uk/debian stable main
\end{lstlisting}

\subsection{Unix}

As Unix and GNU/Linux are very closely related, most versions of Unix will also run OPUS with ease. Again, most modern Unix versions
provide packaged versions of the requirements, or most of them.

\subsection{Microsoft Windows}

At this time Windows cannot be recommended for running OPUS since
\begin{itemize}
  \item in \emph{very} rare places OPUS calls standard unix style utilities; in particular it uses \texttt{grep} and \texttt{tail} in its log
    file viewer. These tools can be obtained for the Windows platform however;
  \item OPUS runs scheduled jobs using \texttt{cron} to perform regular maintainence, you will need to trigger these with the various
    scheduling tools available on the specific version of Windows you are running (see \texttt{at}) for example;
  \item OPUS has been designed and run on Linux and Solaris, at present the various installation scripts and snippets we have produced
    are aimed at such platforms. However, please \emph{do} feel free to send patches with better Windows support.
\end{itemize}

This is \emph{not} to say that OPUS cannot be run on Windows, and if you have expertise with Windows you will have no problems overcoming these
hurdles by yourself.

\section{Manual Installation}

If you plan to install OPUS manually, here is some guidance on the important directories you need to manage.

\begin{description}
  \item[include] contains vital files that the ``main'' scripts of OPUS will use. This should \emph{not} be installed in the web server document root, since
  it also contains sensitive configuration. However, the web server will need read access to these files.
  \item[html] contains the ``main'' scripts and content that OPUS provides. This will need to be in the web server tree.
  \item[templates] contains Smarty templates for output, like include, it should be outside the document root, but accessible to the web server.
  \item[cron] contains PHP scripts that perform automatic maintainence. 
  \item[docs] contains files to produce this manual, not required on a production server.
  \item[etc] contains an example Apache configuration file, suitable for Debian.
  \item[sql\_patch] contains SQL files for importing the intial schema and data into MySQL.
\end{description}

In addition, you should create two more directories, in the same directory as the \lstinline!templates! directory.

\begin{description}
  \item[templates\_c] used by smarty to compile templates, note it \emph{must} be writable by the web server process.
  \item[templates\_cache] used by smarty for various caching, again, must be writable by the web server.
\end{description}

To set the correct permissions you could use commands like:

\begin{lstlisting}{language=bash}
chown root:www-data templates_*
chmod 770 templates_*
\end{lstlisting}

where \lstinline!www-data! is the username for the web server process.

One easy way to deal with this is to copy the whole set into a location \emph{not} under the web server tree, and configuring the web server to have
acess to the html directory only. For Apache, this is more or less done with the file in the etc directory. You will note it also changes the PHP include
path to use the include path, as well as a Smarty path. You might need to change paths accordingly.

\subsection{Configuring your Web Server}

You will now need to ensure that your web server can ``see'' the directories it will need access to. There are several ways to achieve this.

\subsubsection{Directly copy under the web server root}

Perhaps the simplest approach is to copy the \lstinline!html! directory under the web server root. It is \emph{not} recommended that any other
paths are copied in this way for security reasons. This is probably simple, but tedious in the long run since it will make upgrades more difficult.

\subsubsection{Use a symbolic link}

If you are using an operating system that supports symbolic links, you could link the \lstinline!html! directory to a path under the main web server
document root.

\subsubsection{Using a configuration file}

Probably the most flexible and elegant way it to use a configuration file. If you are using Apache 2 you will find a suitable configuration file in the
\lstinline!etc! directory of the OPUS package. Simply copy it to your appropriate configuration directory for apache (probably something like
\lstinline!/etc/apache2/conf.d/! and edit to requirements. Then restart the web server.

\opustip{You will need to ensure that PHP includes the appropriate directories from OPUS and Smarty in its local include path. Although you can do this
by editing the global \lstinline!php.ini! file, it is easier to do this as a local override. The configuration file does this for you. If you use an alternative
method you will almost certainly need to include a file (like \lstinline!.htaccess! for Apache) that overrides this path.
\emph{make sure your \lstinline!include_path! does not include the current directory (.)}, or if
it does, ensure that this is included last on the include paths.}

\subsection{Database Creation}

You should now create a new database. Using your mysql client or otherwise. To use the client, launch it from the
command line like so. If you have (correctly) configured a password for the root user add \lstinline! -p ! to this command.
\begin{lstlisting}[language=bash]
mysql -u root
\end{lstlisting}
or using a password as required. At the client prompt, create a new database
\begin{lstlisting}[language=SQL]
create database opus;
grant all on opus.* to opus_user@localhost
  identified by 'database_password';
\end{lstlisting}

You will need to modify this if the database is on a remote host accordingly. At this point
you can import the schema files from the sql\_patch directory.
\begin{lstlisting}[language=SQL]
source sql_patch/schema.sql
source sql_patch/data.sql
\end{lstlisting}

\subsection{Logging}

OPUS uses substantial logging throughout, and maintains several log files. You should ensure that you have a directory created to which
the web server has write access. See the discussion for the templates directory above. This can be anywhere, such as \lstinline!/var/log/opus/!.

\chapter{Initial Configuration}

After installation, it is usually necessary to copy the file \lstinline!config.php.dist! to
\lstinline!config.php! (in the \lstinline!include! directory) and customize its values.\footnote{Debian installation will take care of this aspect.}

\chapter{Administrators \& Policies}
\index{security}
\index{access}
\index{administrators}
\index{policies}

You should login with the newly installed ``Super-Admin'' user account. This
user has no security restrictions imposed on it, and it is unwise to use it
for day to day purposes. Therefore you will want to make one or more administrator
accounts.

\opustip{The initial login credentials will be \lstinline!admin! and \lstinline!password! so it
is obviously absolutely essential that you change the password as soon as possible,
and certainly before importing other data.}

Normal administrator accounts are bound by \emph{security policies}. These define
the range of actions that may be undertaken by the user. In addition, it is usually
the case that administrators are only able to undertake actions on specific courses
or schools.

\section{Policies}
\index{policies}

You should begin by considering your security policies. By clicking on
\opusacademicmenu{Configuration -> Admins} and then the \opusacademicmenu{Policies} tab,
you will have the opportunity to define a new policy.

Choose a name that reflects your policy and submit it. You are now able to
edit the permissions for that policy. One of the first options is a priority
for use in the help directory (see~\ref{}). The lower the priority, the 
``closer'' this group is to the students, and therefore the
higher up such users will appear in the dynamic help directory.

You may now check and uncheck boxes as you wish to allocate permissions. 
In general, the recommendations are that
\begin{itemize}
  \item Very few, if any, users should be able to delete items - it damages
    the ability to use OPUS as an audit tool.
  \item Few users should be able to create items - these users should be well
    trained and able to act coherently as a group.
  \item Give day to day administrators more abilities to edit than create or delete.
  \item It is possible to create policies that can ``look but not touch'' for those
    users who only need read access to data.
\end{itemize}

In general a hierarchy of at least a couple of policies is recommended, to put
more experienced users in the higher bracket.

There is a default policy for course directors, who have very limited access to
some administrative features. Many advanced features cannot be enabled
for this category.

\section{Administrator Users}

Now you have created suitable policies, you are ready to create administrator
users and allocate them to policies. A policy provides a ceiling on the powers
of a user - it is possible to curtail them further as needed (see~\ref{}).

Clicking on the \opusacademicmenu{Configuration -> Admins} tab will allow you to list and
create new administrator users. This is a two step process.

\subsection{Creating Administrators}

Use the form at the bottom of the page to create a new administrator account.
Most of the fields are self explanatory, but there are some things to note.

\subsubsection{Staff Number}

If the staff number is defined, it will be possible to login using a ``standard''
single sign-on method. Otherwise the username and password generated must be used.

\subsubsection{Options}

It is possible to define if this user will be shown in the Help Directory. This
should normally be yes unless this user will not answer enquiries about students etc..
\opustip{You can uncheck this if you are on leave to cause the
help directory to automatically work around the absence.}

\subsubsection{Username \& Real name}

Generally you should leave this to ``auto'' and let OPUS work them out for you.
On successful creation of a new user OPUS will email them their credentials
(in an email apparently from you).

\subsection{Allocating Activities \& Policies}

Administrator users so created will initially have no security policy defined. This
renders them virtually useless. You must then locate them in the Administrator listing
and \opustext{Edit} them. 

\subsubsection{Policies}
\index{policies}

You can now allocate a security policy for the user. This action will require
confirmation and will be logged in the security log. Any subsequent change of policy
will also require this.

\subsubsection{Activities}
\index{activities}
\index{help!directory}

Allocating activities to an administrator indicates that they are a useful point of
contact for companies who operate in that activity, and will be used in formulating
the Help Directory.

\subsubsection{Root Users}
\index{root users}


If you are logged in as a super-admin or root user you will also be presented with
an option to promote a ``normal'' administrator user to a ``root'' user and vice versa.
You should generally keep \emph{very few} root accounts on your system.

\subsection{Editing User Details}

If you are a normal administrator you will only have permission to edit your own
account, and will not see the details of other accounts.

\chapter{Courses, Groups \& Channels}

Before students are added to the system, it is import to configure schools and
courses. In addition, the concepts of groups and channels can help you deal
with courses in an efficient manner later, so it is worth giving some thought
to that now.

\section{Groups}

OPUS groups courses in two different ways: CV groups and Assessment groups.

A group provides a simple mechanism for dealing with a large number of courses
in one way. Groups can contain dozens of courses, or a single course that
requires special handling.

OPUS installs with a default group for CVs and a default group for Assessments;
if a course is not explicitly added to a group it is assumed to be in these
groups.

\subsection{CV Groups}
\label{CVGroups}

A CV group defines which CVs can be used by students on these courses.

\subsection{Assessment Groups}

An assessment group collects all the courses that are assessed in the same way.
You can define an assessment regime for each of these groups.

\section{Schools}
\index{schools}

A school is simply a division that contains a number of courses. You should 
create one or more schools. For each school it is possible to define its name
and any webpage it uses.

When editing a school record it is possible to label it as \opustext{archived}. Generally,
within OPUS data should not be deleted, since this destroys the audit capabilities
of the system. Labelling an old school as archived removes it from normal visibility
however.

You may also be able to define which administrators have permissions related to this
school. You can choose any administrator and a security policy. Normally you will
wish to use the default policy for the user, but it is also possible to give
curtailed access to a particular school by using a lower security policy.

\opustip{The default policy for a user always provides a ceiling on their abilities.
You cannot use a local policy to increase priviledges, only to reduce them.}

\section{Courses}

Editing courses is much like that of schools. When creating courses if only the
course code is specified OPUS attempts to retrieve the course details from your
SRS, otherwise you may supply full details.

\subsection{Administration Rights}

Just as for schools it is possible to define what administrators have powers over
this course.

\opustip{An administrator attached to the school this course is in already has
permissions for this course. Only use this for administrators who only have
access to one or two courses.}

\subsection{Course Directors}

You can also define a member of academic staff who is the course director. This
member of staff will obtain slighly augmented abilities (generally access to the
student directory (see~\ref{}) and most of its abilities for their students).

\subsection{CV Groups}

You can select a single CV group that this course belongs to. This will define
which CVs can be used by students on that course, and whether they require vetting.

\subsection{Assessment Groups}

A course is likely to belong to different assessment groups as the years progress and
assessment regimes change. In OPUS, this is handled by creating a new assessment group
and recording that students in subsequent years are members of the new group.

For each assessment group a course is allocated into a start and end year may be
specified. If for example the start year is 2006 and there is no end year then all
students seeking placement in 2006 or later belong to this group. Similarly an entry
showing no start year and an end year of 2005 indicates that students seeking
placement in 2005 and before belong to that group. Obviously it is possible to
specify both a start and end year.

\chapter{Importing Students}

Importing students is a vital step in populating OPUS with the data you need. There are several
approaches to this.

\section{Import from SRS}
\index{web services}

It is possible, with an appropriate web services layer for your student records system to import
students on mass. The form from \opusacademicmenu{Configuration -> Import Data} will allow
you to specify which year group you plan to import, and what academic year they will be seeking
placement. If it is during the 2007-2008 year for example, you should specify 2007 here. Generally
you should leave other information such as passwords blank, and normally you will wish to
import students with a status of ``Required'' indicating that they are seeking placement.

You will note a checkbox indicating that the import is a test. Try the import with the checkbox
ticked first of all and carefully ensure that everything went well, before removing the tick
to do the import for real.

You can repeatedly import the same students over and over again and OPUS will only import
any new students it has not seen before.

\section{From CSV files}

The University of Ulster uses CSV files to store student lists that can also be used for this
purpose, by browsing to the file first.

TODO: Specify the file format here.

\section{Individual Students}

It is also possible to add students individually using an option available from the Student
Directory.

\chapter{Student Directory}

The student directory is perhaps the most important administrative feature of OPUS. It provides
a mechanism to obtain or add information on individual students, and to extract information from
whole classes of students.

\section{Searching}

\subsection{Simple Searches}

\subsection{Advanced Searches}

\subsection{Communication}

\section{Student Records}

When one has select a student record, there are five tabs that allow you to handle it.

\subsection{CVs}
\index{CVs}

This tab provides information about the CVs that a student has completed, and whether or not they can
be used to apply for vacancies. It also provides the mechanism by which a CV can be approved for use
(or have that approval revoked) if this feature is enabled.

The CV group (see~\ref{CVGroups}) that the student belongs to is given, so that any errors can be seen. Note that
it is not possible to explictly change the CV Group for a single student, since the grouping is done based on course code.

\subsection{Channels}
\index{channels}

The channels which the student is automatically subscribed to are shown here. Any errors should be handled by correcting the
configuration of the channel itself (see~\ref{}).

\subsection{Companies}

This tab shows all the applications for the student record. It is possible to remove any application here, which should be done with caution
since companies may wonder why this has happened. Note that students do \emph{not} have the ability  to retract an application for this very reason.

If a student has achieved placement with one of these companies the appropriate link can be clicked to begin the process of recording the placement.

\subsection{Status}

This is the most important tab, allowing most of the information that is of day-to-day use to be viewed and edited.

\subsection{Notes}

\section{Management Statistics}

%
%
%

\chapter{Company Directory}

\section{Searching for companies}

\section{Adding and updating company records}

The company directory has an option to create a new company.

\opustip{It is vitally important you check if a company already exists in the database
before you create what might be a duplicate record.}

\section{Contacts}

\section{Resources}

\section{Companies and channels}

%
%
%

\chapter{Contact Directory}

\section{Searching for contacts}

\section{Managing contacts}

\section{Contacts and channels}

%
%
%

\chapter{Staff Directory}

\section{Searching for staff}

\section{Managing staff}

\section{Staff and students}

\section{Staff and channels}

%
%
%

\chapter{Help, Templates \& Resources}

\section{Help}

\subsection{Creating help}

\subsection{Editing help}

\section{Templates}

\subsection{Creating templates}

\subsection{Editing templates}

\section{Resources}

\subsection{Adding a resource}

\subsection{Resource information}

\subsection{Modifying a resource}

%
%
%

\chapter{Logs \& Status}

\section{Log files}

\subsection{Searching the log files}

\section{Status}

%
%
%

\chapter{Assessment}

\section{Assessments}

\subsection{Creating an assessment record}

\subsection{Creating an assessment skin}

\section{Creating an assessment regime}

%
%
%

\part{Academic Tutor Guide}

\chapter{Academic Tutors}

\part{Company HR Guide}

\chapter{Using OPUS to hire students}

\part{Workplace Supervisor Guide}

\chapter{Using OPUS to help supervise students}

\part{Student Guide}

%
%
%

\chapter{OPUS and PDP}

\section{PDP}

\section{The PDSystem}

%
%
%

\chapter{Preparing for and Getting a Placement}

\section{CVs}

\section{Covering Letters}

\section{e-Portfolios}

\section{Searching for vacancies}

\section{Applying for vacancies}

\section{Managing applications}

\section{Health \& Safety}

\section{Self assessments}

%
%
%


\chapter{Using OPUS while on placement}

\section{Announcements}

\section{Resources}

\section{Assessment}

%
%
%

\chapter{OPUS after your placement}

\section{Assessment}

\section{Learning the lessons}

%
% Index
%

\printindex
\markboth{Index}{}
\addcontentsline{toc}{chapter}{Index}

\end{document}



