%
% OPUS Manual
%

% Latex 2e

\pdfcompresslevel=9
\pdfoutput=1

% Try 12 point
\documentclass[12 pt]{book}
%
% Pick one of the following font packages (if you don't like CM)
%
%\usepackage{pslatex}
\usepackage{times}
%\usepackage{palatino}
%\usepackage{newcent}
%
%
\usepackage{hyperref}
\usepackage{makeidx}
\usepackage{fancyhdr}
\usepackage[pdftex]{graphicx}
%\usepackage{amsfonts}
\usepackage{url}
\usepackage{color}
\usepackage{layout}
\usepackage[margin=3cm,innermargin=1.5cm,outermargin=1cm,includemp=true,marginparsep=1 cm, marginparwidth=2.5 cm, paperwidth=190 mm,paperheight=260 mm]{geometry}
\usepackage{ifthen}
%
% Define some dingbats
%
\usepackage{pifont}

%
% Change the default bullet
%
\renewcommand{\labelitemi}{%
        \raisebox{-.25ex}{\ding{43}}}

%
% Does it all have to be Arial now? <sigh>
%
\renewcommand{\familydefault}{\sfdefault}

\DeclareGraphicsExtensions{.pdf,.png,.mps}

\pdfinfo{
  /Title (OPUS Manual)
  /Author (Dr Colin Turner)
  /Producer (PDFLaTeX)
}

\makeindex

\begin{document}

% New commands, and notation
%
% Preamble for all documents
%
% Colin Turner
%

% Latex 2e

\newcommand{\UniversityName}{the University of Ulster}
\newcommand{\ShortUniversityName}{Ulster}
\newcommand{\OpusUrl}{http://opus.ulster.ac.uk}
\newcommand{\DevelopmentUrl}{http://foss.ulster.ac.uk/projects/opus/}
\newcommand{\ContactEmail}{opus@foss.ulster.ac.uk}

% Headings
%
% Headings for notes
%
% Colin Turner
%

% Latex 2e

% Headings



%
% New fancyhdr version
%
\pagestyle{fancy}
\addtolength{\headwidth}{\marginparsep}
\addtolength{\headwidth}{\marginparwidth}
\renewcommand{\chaptermark}[1]{\markboth{#1}{}}
\renewcommand{\sectionmark}[1]{\markright{\thesection\ #1}}
\fancyhf{}
\renewcommand{\headrulewidth}{0pt} % get rid of the line
\fancyhead[LE,RO]{\colorbox{cyan}{\color{white}\thepage\normalcolor}}
\fancyhead[LO]{\colorbox{cyan}{\color{white}\rightmark\normalcolor}}
\fancyhead[RE]{\colorbox{cyan}{\color{white}\leftmark\normalcolor}}
\fancypagestyle{plain}{%
\fancyhead{} % get rid of headers
\renewcommand{\headrulewidth}{0pt} % and the line
}

%
% Now some LaTeX magic from
% http://research.cmis.csiro.au/gjw/tex/docs/fancyhdr.pdf
% to make extra pages blank
%


\makeatletter
\def\cleardoublepage{\clearpage\if@twoside \ifodd\c@page\else
\hbox{}
\vspace*{\fill}
%\begin{center}
%This page intentionally contains only this sentence.
%\end{center}
\vspace{\fill}
\thispagestyle{empty}
\newpage
\if@twocolumn\hbox{}\newpage\fi\fi\fi}
\makeatother



% New environments

\newcommand{\pdstext}[1]{\color{magenta}{\sffamily\bfseries #1}\normalcolor} 
\newcommand{\opustext}[1]{\color{magenta}{\sffamily\bfseries #1}\normalcolor} 
\newcommand{\pdsstudentmenu}[1]{\color{blue}{\ttfamily \bfseries #1}\normalcolor} 
\newcommand{\pdsacademicmenu}[1]{\color{red}{\ttfamily \bfseries #1}\normalcolor} 
\newcommand{\opusacademicmenu}[1]{\color{red}{\ttfamily \bfseries #1}\normalcolor} 
\newcommand{\pdsiconcd}{\marginpar{\includegraphics[scale=1.0]{png/icon_cd.png}}}
\newcommand{\pdsiconsa}{\marginpar{\includegraphics[scale=1.0]{png/icon_sa.png}}}
\newcommand{\pdsiconmc}{\marginpar{\includegraphics[scale=1.0]{png/icon_mc.png}}}
\newcommand{\pdsicons}{\marginpar{\includegraphics[scale=1.0]{png/icon_s.png}}}
\newcommand{\pdsiconpt}{\marginpar{\includegraphics[scale=1.0]{png/icon_pt.png}}}
\newcommand{\pdsbookletref}[1]
{\marginpar
  {\makebox[0 pt][l]
    {\includegraphics[scale=1.0]{png/icon_book.png}
  }
  \parbox{2 cm}{{\sffamily \bfseries \tiny #1}}}}

\newcommand{\pdstip}[1]{{\bfseries TIP!} \emph{#1}}
\newcommand{\opustip}[1]{{\bfseries TIP!} \emph{#1}}

%
% Page Dimensions, required to be custom for this job
%
\setlength{\paperheight}{260 mm}
\setlength{\paperwidth}{190 mm}
\setlength{\pdfpageheight}{\paperheight}
\setlength{\pdfpagewidth}{\paperwidth}


\title{OPUS Manual}

\author{Dr Colin Turner\\\url{c.turner@ulster.ac.uk}}

\date{October, 2006}

%%% ----------------------------------------------------------------------

\pagenumbering{roman}
\maketitle

\pagestyle{fancy}

\tableofcontents
\markboth{Contents}{Contents}

%\listoftables

%\listoffigures

%
% Introduction, unnumbered chapter
%

\setlength{\parindent}{0 cm}
\setlength{\parskip}{0.2 cm}

\chapter*{Introduction\markboth{Introduction}{}}
Welcome to the OPUS manual. As you will see this is really a work very much in progress.

%\newpage
%\pagecolor{red}
%\pagecolor{white}
%\newpage

\chapter{How to use this workbook}
\pagenumbering{arabic}

You 
might choose to read this workbook from cover to cover,
or to dip into it as required. The following conventions
have been 
used to help you get the most of this book whatever way you choose to use it. 

\section{Signposts}
\index{Signposts}

At various points in the workbook, signposts are used in
the margins to indicate that a section is of specific
interest to a particular role you may have.

\parbox{2 cm}{\includegraphics{png/icon_cd.png}}
\parbox{9 cm}{Course Directors ({\bfseries CD}) have many varied responsibilities,
including often acting as an Adviser of Studies.}
\index{Course Director}
\medskip

\parbox{2 cm}{\includegraphics{png/icon_sa.png}}
\parbox{9 cm}{Studies Advisers ({\bfseries SA}). The PDSystem has a great deal of
dedicated support for the studies advice process.}
\index{adviser of study}
\index{studies adviser|see{adviser of study}}
\medskip

\parbox{2 cm}{\includegraphics{png/icon_mc.png}}
\parbox{9 cm}{Module Coordinators ({\bfseries MC}). Many
of the facilities useful to Course Directors are also
useful to Module Coordinators.}
\index{Module Coordinators}
\medskip

\parbox{2 cm}{\includegraphics{png/icon_s.png}}
\parbox{9 cm}{Supervisors ({\bfseries S}). For the purposes
of this workbook, supervisors are taken to include those
responsible for undergraduate projects, right up to PhD
supervision.}
\index{Supervising}
\medskip

\parbox{2 cm}{\includegraphics{png/icon_pt.png}}
\parbox{9 cm}{Placement Tutors ({\bfseries 
PT}). The PDSystem provides many possibilities for
monitoring and assessing placements, and integrates into
OPUS. OPUS is another University of Ulster on-line system
which deals comprehensively with the whole
placement process: it aids in the tasks of finding students
a placement; recording their placement information,
assessing their placement and more. More details can
be found at \url{http://opus.ulster.ac.uk}.}
\index{placement!tutors}

\section{Typographical Conventions}
\index{typographical conventions}

Text which appears on OPUS screens will be
presented \pdstext{like this}.

Where specific menu selections on OPUS are detailed they will
be shown as follows: 
\begin{itemize}
\item \pdsstudentmenu{Section -> Subsection} when the student strand is discussed; and
\item \pdsacademicmenu{Section -> Subsection} when the staff strand is being discussed.
\end{itemize}

\pdstip{Sometimes an important tip will appear like this}.

\part{Administration Guide}

\chapter{Installation}

\chapter{Initial Configuration}

After installation, it is necessary to copy the file config.php.dist to
config.php and customize its values.

\chapter{Administrators \& Policies}
\index{security}
\index{access}
\index{administrators}
\index{policies}

You should login with the newly installed ``Super-Admin'' user account. This
user has no security restrictions imposed on it, and it is unwise to use it
for day to day purposes. Therefore you will want to make one or more administrator
accounts.

Normal administrator accounts are bound by \emph{security policies}. These define
the range of actions that may be undertaken by the user. In addition, it is usually
the case that administrators are only able to undertake actions on specific courses
or schools.

\section{Policies}
\index{policies}

You should begin by considering your security policies. By clicking on
\opusacademicmenu{Edit Admins} and then the \opusacademicmenu{Policies} tab,
you will have the opportunity to define a new policy.

Choose a name that reflects your policy and submit it. You are now able to
edit the permissions for that policy. One of the first options is a priority
for use in the help directory (see~\ref{}). The lower the priority, the 
``closer'' this group is to the students, and therefore the
higher up such users will appear in the dynamic help directory.

You may now check and uncheck boxes as you wish to allocate permissions. 
In general, the recommendations are that
\begin{itemize}
  \item Very few, if any, users should be able to delete items - it damages
    the ability to use OPUS as an audit tool.
  \item Few users should be able to create items - these users should be well
    trained and able to act coherently as a group.
  \item Give day to day administrators more abilities to edit than create or delete.
  \item It is possible to create policies that can ``look but not touch'' for those
    users who only need read access to data.
\end{itemize}

In general a hierarchy of at least a couple of policies is recommended, to put
more experienced users in the higher bracket.

There is a default policy for course directors, who have very limited access to
some administrative features. Many advanced features cannot be enabled
for this category.

\section{Administrator Users}

Now you have created suitable policies, you are ready to create administrator
users and allocate them to policies. A policy provides a ceiling on the powers
of a user - it is possible to curtail them further as needed (see~\ref{}).

Clicking on the \opusacademicmenu{Admins} tab will allow you to list and
create new administrator users. This is a two step process.

\subsection{Creating Administrators}

Use the form at the bottom of the page to create a new administrator account.
Most of the fields are self explanatory, but there are some things to note.

\subsubsection{Staff Number}

If the staff number is defined, it will be possible to login using a ``standard''
single sign-on method. Otherwise the username and password generated must be used.

\subsubsection{Options}

It is possible to define if this user will be shown in the Help Directory. This
should normally be yes unless this user will not answer enquiries about students etc..
\opustip{You can uncheck this if you are on leave to cause the
help directory to automatically work around the absence.}

\subsubsection{Username \& Real name}

Generally you should leave this to ``auto'' and let OPUS work them out for you.
On successful creation of a new user OPUS will email them their credentials
(in an email apparently from you).

\subsection{Allocating Activities \& Policies}

Administrator users so created will initially have no security policy defined. This
renders them virtually useless. You must then locate them in the Administrator listing
and \opustext{Edit} them. 

\subsubsection{Policies}
\index{policies}

You can now allocate a security policy for the user. This action will require
confirmation and will be logged in the security log. Any subsequent change of policy
will also require this.

\subsubsection{Activities}
\index{activities}
\index{help!directory}

Allocating activities to an administrator indicates that they are a useful point of
contact for companies who operate in that activity, and will be used in formulating
the Help Directory.

\subsubsection{Root Users}
\index{root users}


If you are logged in as a super-admin or root user you will also be presented with
an option to promote a ``normal'' administrator user to a ``root'' user and vice versa.
You should generally keep \emph{very few} root accounts on your system.

\subsection{Editing User Details}

If you are a normal administrator you will only have permission to edit your own
account, and will not see the details of other accounts.

\chapter{Courses, Groups \& Channels}

Before students are added to the system, it is import to configure schools and
courses. In addition, the concepts of groups and channels can help you deal
with courses in an efficient manner later, so it is worth giving some thought
to that now.

\section{Groups}

OPUS groups courses in two different ways: CV groups and Assessment groups.

A group provides a simple mechanism for dealing with a large number of courses
in one way. Groups can contain dozens of courses, or a single course that
requires special handling.

OPUS installs with a default group for CVs and a default group for Assessments;
if a course is not explicitly added to a group it is assumed to be in these
groups.

\subsection{CV Groups}

A CV group defines which CVs can be used by students on these courses.

\subsection{Assessment Groups}

An assessment group collects all the courses that are assessed in the same way.
You can define an assessment regime for each of these groups.

\section{Schools}
\index{schools}

A school is simply a division that contains a number of courses. You should 
create one or more schools. For each school it is possible to define its name
and any webpage it uses.

When editing a school record it is possible to label it as \opustext{archived}. Generally,
within OPUS data should not be deleted, since this destroys the audit capabilities
of the system. Labelling an old school as archived removes it from normal visibility
however.

You may also be able to define which administrators have permissions related to this
school. You can choose any administrator and a security policy. Normally you will
wish to use the default policy for the user, but it is also possible to give
curtailed access to a particular school by using a lower security policy.

\opustip{The default policy for a user always provides a ceiling on their abilities.
You cannot use a local policy to increase priviledges, only to reduce them.}

\section{Courses}

Editing courses is much like that of schools. When creating courses if only the
course code is specified OPUS attempts to retrieve the course details from your
SRS, otherwise you may supply full details.

\subsection{Administration Rights}

Just as for schools it is possible to define what administrators have powers over
this course.

\opustip{An administrator attached to the school this course is in already has
permissions for this course. Only use this for administrators who only have
access to one or two courses.}

\subsection{Course Directors}

You can also define a member of academic staff who is the course director. This
member of staff will obtain slighly augmented abilities (generally access to the
student directory (see~\ref{}) and most of its abilities for their students).

\subsection{CV Groups}

You can select a single CV group that this course belongs to. This will define
which CVs can be used by students on that course, and whether they require vetting.

\subsection{Assessment Groups}

A course is likely to belong to different assessment groups as the years progress and
assessment regimes change. In OPUS, this is handled by creating a new assessment group
and recording that students in subsequent years are members of the new group.

For each assessment group a course is allocated into a start and end year may be
specified. If for example the start year is 2006 and there is no end year then all
students seeking placement in 2006 or later belong to this group. Similarly an entry
showing no start year and an end year of 2005 indicates that students seeking
placement in 2005 and before belong to that group. Obviously it is possible to
specify both a start and end year.

\chapter{Importing Students}

\chapter{Student Directory}

\chapter{Company Directory}

\chapter{Contact Directory}

\chapter{Staff Directory}

\chapter{Help, Templates \& Resources}

\chapter{Logs \& Status}

\chapter{Assessments}

\part{Academic Tutor Guide}

\part{Company HR Guide}

\part{Workplace Supervisor Guide}

\part{Student Guide}

%
% Index
%

\printindex
\markboth{Index}{}
\addcontentsline{toc}{chapter}{Index}

\end{document}



