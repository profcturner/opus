%
% OPUS Manual
%

% Latex 2e

\pdfcompresslevel=9
\pdfoutput=1

% Try 12 point
\documentclass[12 pt]{book}
%
% Pick one of the following font packages (if you don't like CM)
%
%\usepackage{pslatex}
\usepackage{times}
%\usepackage{palatino}
%\usepackage{newcent}
%
%
\usepackage{hyperref}
\usepackage{makeidx}
\usepackage{fancyhdr}
\usepackage[pdftex]{graphicx}
%\usepackage{amsfonts}
\usepackage{url}
\usepackage{color}
\usepackage{listings}
\usepackage{layout}
\usepackage[margin=3cm,innermargin=1.5cm,outermargin=1cm,includemp=true,marginparsep=1 cm, marginparwidth=2.5 cm, paperwidth=190 mm,paperheight=260 mm]{geometry}
\usepackage{ifthen}
%
% Define some dingbats
%
\usepackage{pifont}

%
% Change the default bullet
%
\renewcommand{\labelitemi}{%
        \raisebox{-.25ex}{\ding{43}}}

%
% Does it all have to be Arial now? <sigh>
%
\renewcommand{\familydefault}{\sfdefault}

\DeclareGraphicsExtensions{.pdf,.png,.mps}

\pdfinfo{
  /Title (OPUS Manual)
  /Author (Dr Colin Turner)
  /Producer (PDFLaTeX)
}

\makeindex

\begin{document}

% New commands, and notation
%
% Preamble for all documents
%
% Colin Turner
%

% Latex 2e

\newcommand{\UniversityName}{the University of Ulster}
\newcommand{\ShortUniversityName}{Ulster}
\newcommand{\OpusUrl}{http://opus.ulster.ac.uk}
\newcommand{\DevelopmentUrl}{http://foss.ulster.ac.uk/projects/opus/}
\newcommand{\ContactEmail}{opus@foss.ulster.ac.uk}

% Headings
%
% Headings for notes
%
% Colin Turner
%

% Latex 2e

% Headings



%
% New fancyhdr version
%
\pagestyle{fancy}
\addtolength{\headwidth}{\marginparsep}
\addtolength{\headwidth}{\marginparwidth}
\renewcommand{\chaptermark}[1]{\markboth{#1}{}}
\renewcommand{\sectionmark}[1]{\markright{\thesection\ #1}}
\fancyhf{}
\renewcommand{\headrulewidth}{0pt} % get rid of the line
\fancyhead[LE,RO]{\colorbox{cyan}{\color{white}\thepage\normalcolor}}
\fancyhead[LO]{\colorbox{cyan}{\color{white}\rightmark\normalcolor}}
\fancyhead[RE]{\colorbox{cyan}{\color{white}\leftmark\normalcolor}}
\fancypagestyle{plain}{%
\fancyhead{} % get rid of headers
\renewcommand{\headrulewidth}{0pt} % and the line
}

%
% Now some LaTeX magic from
% http://research.cmis.csiro.au/gjw/tex/docs/fancyhdr.pdf
% to make extra pages blank
%


\makeatletter
\def\cleardoublepage{\clearpage\if@twoside \ifodd\c@page\else
\hbox{}
\vspace*{\fill}
%\begin{center}
%This page intentionally contains only this sentence.
%\end{center}
\vspace{\fill}
\thispagestyle{empty}
\newpage
\if@twocolumn\hbox{}\newpage\fi\fi\fi}
\makeatother


% Listings package
\lstset{
  language=PHP,
  alsolanguage=HTML,
  basicstyle=\ttfamily,
  flexiblecolumns=true,
  lineskip=-2pt,
  commentstyle=\color{blue},
  morecomment=[s][\color{red}]{/**}{*/},
%  frame=l,
  numbers=left,
  numberstyle=\tiny,
  stepnumber=5,
  numbersep=5pt,
  backgroundcolor=\color[rgb]{0.95, 0.95, 0.95}
}


% New environments

\newcommand{\pdstext}[1]{\color{magenta}{\sffamily\bfseries #1}\normalcolor} 
\newcommand{\opustext}[1]{\color{magenta}{\sffamily\bfseries #1}\normalcolor} 
\newcommand{\pdsstudentmenu}[1]{\color{blue}{\ttfamily \bfseries #1}\normalcolor} 
\newcommand{\pdsacademicmenu}[1]{\color{red}{\ttfamily \bfseries #1}\normalcolor} 
\newcommand{\opusacademicmenu}[1]{\color{red}{\ttfamily \bfseries #1}\normalcolor} 
\newcommand{\pdsiconcd}{\marginpar{\includegraphics[scale=1.0]{png/icon_cd.png}}}
\newcommand{\pdsiconsa}{\marginpar{\includegraphics[scale=1.0]{png/icon_sa.png}}}
\newcommand{\pdsiconmc}{\marginpar{\includegraphics[scale=1.0]{png/icon_mc.png}}}
\newcommand{\pdsicons}{\marginpar{\includegraphics[scale=1.0]{png/icon_s.png}}}
\newcommand{\pdsiconpt}{\marginpar{\includegraphics[scale=1.0]{png/icon_pt.png}}}
\newcommand{\pdsbookletref}[1]
{\marginpar
  {\makebox[0 pt][l]
    {\includegraphics[scale=1.0]{png/icon_book.png}
  }
  \parbox{2 cm}{{\sffamily \bfseries \tiny #1}}}}

\newcommand{\pdstip}[1]{{\bfseries TIP!} \emph{#1}}
\newcommand{\opustip}[1]{{\bfseries TIP!} \emph{#1}}

%
% Page Dimensions, required to be custom for this job
%
\setlength{\paperheight}{260 mm}
\setlength{\paperwidth}{190 mm}
\setlength{\pdfpageheight}{\paperheight}
\setlength{\pdfpagewidth}{\paperwidth}


\title{OPUS Manual}

\author{Dr Colin Turner\\\url{c.turner@ulster.ac.uk}}

\date{October, 2006}

%%% ----------------------------------------------------------------------

\pagenumbering{roman}
\maketitle

\pagestyle{fancy}

\tableofcontents
\markboth{Contents}{Contents}

%\listoftables

%\listoffigures

%
% Introduction, unnumbered chapter
%

\setlength{\parindent}{0 cm}
\setlength{\parskip}{0.2 cm}

\chapter*{Introduction\markboth{Introduction}{}}
Welcome to the OPUS manual. As you will see this is really a work very much in progress. Please note that the most up-to-date version will always be
found at the development site for OPUS at \url{http://foss.ulster.ac.uk/projects/opus}, currently only in source repository, but hopefully we will enable
automatic builds of the documentation in the near future.

%\newpage
%\pagecolor{red}
%\pagecolor{white}
%\newpage

\chapter{How to use this workbook}
\pagenumbering{arabic}

You 
might choose to read this workbook from cover to cover,
or to dip into it as required. The following conventions
have been 
used to help you get the most of this book whatever way you choose to use it. 

\section{Signposts}
\index{Signposts}

At various points in the workbook, signposts are used in
the margins to indicate that a section is of specific
interest to a particular role you may have.

\parbox{2 cm}{\includegraphics{png/icon_cd.png}}
\parbox{9 cm}{Course Directors ({\bfseries CD}) have many varied responsibilities,
including often acting as an Adviser of Studies.}
\index{Course Director}
\medskip

\parbox{2 cm}{\includegraphics{png/icon_sa.png}}
\parbox{9 cm}{Studies Advisers ({\bfseries SA}). The PDSystem has a great deal of
dedicated support for the studies advice process.}
\index{adviser of study}
\index{studies adviser|see{adviser of study}}
\medskip

\parbox{2 cm}{\includegraphics{png/icon_mc.png}}
\parbox{9 cm}{Module Coordinators ({\bfseries MC}). Many
of the facilities useful to Course Directors are also
useful to Module Coordinators.}
\index{Module Coordinators}
\medskip

\parbox{2 cm}{\includegraphics{png/icon_s.png}}
\parbox{9 cm}{Supervisors ({\bfseries S}). For the purposes
of this workbook, supervisors are taken to include those
responsible for undergraduate projects, right up to PhD
supervision.}
\index{Supervising}
\medskip

\parbox{2 cm}{\includegraphics{png/icon_pt.png}}
\parbox{9 cm}{Placement Tutors ({\bfseries 
PT}). The PDSystem provides many possibilities for
monitoring and assessing placements, and integrates into
OPUS. OPUS is another University of Ulster on-line system
which deals comprehensively with the whole
placement process: it aids in the tasks of finding students
a placement; recording their placement information,
assessing their placement and more. More details can
be found at \url{http://opus.ulster.ac.uk}.}
\index{placement!tutors}

\section{Typographical Conventions}
\index{typographical conventions}

Text which appears on OPUS screens will be
presented \pdstext{like this}.

Where specific menu selections on OPUS are detailed they will
be shown as follows: 
\begin{itemize}
\item \pdsstudentmenu{Section -> Subsection} when the student strand is discussed; and
\item \pdsacademicmenu{Section -> Subsection} when the staff strand is being discussed.
\end{itemize}

\pdstip{Sometimes an important tip will appear like this}.

\part{Administration Guide}

\chapter{Installation}

We look at the various requirements of OPUS here and how to install it.

\section{Requirements}
\index{requirements}

OPUS can be probably be run on top of  any system that supports PHP%
\footnote{A popular web scripting language (\url{http://www.php.net})} and MySQL.%
\footnote{A relational database engine (\url{http://www.mysql.com})}

At the time of writing, specific requirements are
\begin{itemize}
  \item PHP version 5.1 or higher (since PDO is planned to be used in new code);
  \item MySQL version 4 or higher;
  \item Smarty\footnote{A templating enginer (\url{http://smarty.php.net})} version 1.3 or higher;
  \item Pear;%
    \footnote{Extensions to PHP, found at \url{http://pear.php.net}. Pear is required for DB support, and was used for mail\_mime support prior to version 3.3.0,
     but will probably be phased out in version 4.0.0 when built in PDO support is likely to be used.}
  \item Perl version 5 or higher.\footnote{A powerful scripting language, required for some offline maintainence, available from \url{http://www.perl.com}.}
\end{itemize}

\section{Operating Systems}
\index{operating systems}
\index{Debian GNU/Linux}

It should be straight forward to install OPUS on most variants of Unix, or Linux and, with perhaps some more work Microsoft Windows.

\subsection{GNU/Linux}

Various distributions of GNU/Linux, or simply Linux, as they are often called, are well suited to running OPUS. Almost all modern versions
package all or most of the requirements listed above.

The absolutely simplest way is to install OPUS on a server running Debian GNU/Linux 4.0 (Etch) or above%
\footnote{Available for many architectures from \url{http://www.debian.org}.}%
since OPUS is specifically \emph{packaged} for Debian.

Indeed, using the Debian packages will allow you to greatly circumvent many of the other issues in this, and the following chapter, since
much of the installation and packaging will be handled automatically.

To ensure automatic installation and upgrades, add the following line to your configuration for \lstinline!apt! usually found in
\lstinline!/etc/apt/sources.list!.

\begin{lstlisting}
deb http://foss.ulster.ac.uk/debian stable main
\end{lstlisting}

\subsection{Unix}

As Unix and GNU/Linux are very closely related, most versions of Unix will also run OPUS with ease. Again, most modern Unix versions
provide packaged versions of the requirements, or most of them.

\subsection{Microsoft Windows}

At this time Windows cannot be recommended for running OPUS since
\begin{itemize}
  \item in \emph{very} rare places OPUS calls standard unix style utilities; in particular it uses \texttt{grep} and \texttt{tail} in its log
    file viewer. These tools can be obtained for the Windows platform however;
  \item OPUS runs scheduled jobs using \texttt{cron} to perform regular maintainence, you will need to trigger these with the various
    scheduling tools available on the specific version of Windows you are running (see \texttt{at}) for example;
  \item OPUS has been designed and run on Linux and Solaris, at present the various installation scripts and snippets we have produced
    are aimed at such platforms. However, please \emph{do} feel free to send patches with better Windows support.
\end{itemize}

This is \emph{not} to say that OPUS cannot be run on Windows, and if you have expertise with Windows you will have no problems overcoming these
hurdles by yourself.

\section{Manual Installation}

If you plan to install OPUS manually, here is some guidance on the important directories you need to manage.

\begin{description}
  \item[include] contains vital files that the ``main'' scripts of OPUS will use. This should \emph{not} be installed in the web server document root, since
  it also contains sensitive configuration. However, the web server will need read access to these files.
  \item[html] contains the ``main'' scripts and content that OPUS provides. This will need to be in the web server tree.
  \item[templates] contains Smarty templates for output, like include, it should be outside the document root, but accessible to the web server.
  \item[cron] contains PHP scripts that perform automatic maintainence. 
  \item[docs] contains files to produce this manual, not required on a production server.
  \item[etc] contains an example Apache configuration file, suitable for Debian.
  \item[sql\_patch] contains SQL files for importing the intial schema and data into MySQL.
\end{description}

In addition, you should create two more directories, in the same directory as the \lstinline!templates! directory.

\begin{description}
  \item[templates\_c] used by smarty to compile templates, note it \emph{must} be writable by the web server process.
  \item[templates\_cache] used by smarty for various caching, again, must be writable by the web server.
\end{description}

To set the correct permissions you could use commands like:

\begin{lstlisting}{language=bash}
chown root:www-data templates_*
chmod 770 templates_*
\end{lstlisting}

where \lstinline!www-data! is the username for the web server process.

One easy way to deal with this is to copy the whole set into a location \emph{not} under the web server tree, and configuring the web server to have
acess to the html directory only. For Apache, this is more or less done with the file in the etc directory. You will note it also changes the PHP include
path to use the include path, as well as a Smarty path. You might need to change paths accordingly.

\subsection{Configuring your Web Server}

You will now need to ensure that your web server can ``see'' the directories it will need access to. There are several ways to achieve this.

\subsubsection{Directly copy under the web server root}

Perhaps the simplest approach is to copy the \lstinline!html! directory under the web server root. It is \emph{not} recommended that any other
paths are copied in this way for security reasons. This is probably simple, but tedious in the long run since it will make upgrades more difficult.

\subsubsection{Use a symbolic link}

If you are using an operating system that supports symbolic links, you could link the \lstinline!html! directory to a path under the main web server
document root.

\subsubsection{Using a configuration file}

Probably the most flexible and elegant way it to use a configuration file. If you are using Apache 2 you will find a suitable configuration file in the
\lstinline!etc! directory of the OPUS package. Simply copy it to your appropriate configuration directory for apache (probably something like
\lstinline!/etc/apache2/conf.d/! and edit to requirements. Then restart the web server.

\opustip{You will need to ensure that PHP includes the appropriate directories from OPUS and Smarty in its local include path. Although you can do this
by editing the global \lstinline!php.ini! file, it is easier to do this as a local override. The configuration file does this for you. If you use an alternative
method you will almost certainly need to include a file (like \lstinline!.htaccess! for Apache) that overrides this path.
\emph{make sure your \lstinline!include_path! does not include the current directory (.)}, or if
it does, ensure that this is included last on the include paths.}

\subsection{Database Creation}

You should now create a new database. Using your mysql client or otherwise. To use the client, launch it from the
command line like so. If you have (correctly) configured a password for the root user add \lstinline! -p ! to this command.
\begin{lstlisting}[language=bash]
mysql -u root
\end{lstlisting}
or using a password as required. At the client prompt, create a new database
\begin{lstlisting}[language=SQL]
create database opus;
grant all on opus.* to opus_user@localhost
  identified by 'database_password';
\end{lstlisting}

You will need to modify this if the database is on a remote host accordingly. At this point
you can import the schema files from the sql\_patch directory.
\begin{lstlisting}[language=SQL]
source sql_patch/schema.sql
source sql_patch/data.sql
\end{lstlisting}

\subsection{Logging}

OPUS uses substantial logging throughout, and maintains several log files. You should ensure that you have a directory created to which
the web server has write access. See the discussion for the templates directory above. This can be anywhere, such as \lstinline!/var/log/opus/!.







\chapter{Administrators \& Policies}
\index{security}
\index{access}
\index{administrators}
\index{policies}

This chapter is required reading for the top level administrators of OPUS, it
is of less interest to other administrators who should find that things are
already correctly configured for them.

You should login with the newly installed ``Super-Admin'' user account. This
user has no security restrictions imposed on it, and it is unwise to use it
for day to day purposes. Therefore you will want to make one or more ``normal''
administrator accounts.

\opustip{The initial login credentials will be \lstinline!admin! and 
\lstinline!password! so it is obviously absolutely essential that you change 
the password as soon as possible, and certainly before importing other data.}

Normal administrator accounts are bound by \emph{security policies}. These
define the range of actions that may be undertaken by the user. In addition, 
each administrator may be limited in scope so that they may exercise their
powers on an institutional level, or at faculty, school or individual
programme, or even a combination of the above.

\section{Policies}
\index{policies}

You should begin by considering your security policies. By clicking on
\opusacademicmenu{Advanced -> Policies} you will be able to view and manipulate
policies. OPUS already comes configured with a number of policies, that you may
wish to add to, or adjust.

More or less, \emph{Placement Coordinators} operate ata high level
institutionally, should be better trained with OPUS and well versed in your
policies for adding companies. Out of the box, they can add companies and other
details that lesser users can only edit.

By contrast, \emph{Placement Tutors} usually have responsibility for a few
programmes. They have less power, but remember, you can \emph{help} staff by
restricting their access so they can't make foolish errors.

The \emph{Viewer} policy is intended for staff who need read only access to
data but should never make changes.

The \emph{Priority} attached to a policy simply dictates a kind of distance
from the students. The lower the priority, the closer a policy holder is to
the students, and so it is assumed they will be able to help more fully deal
with queries about individual students.

Clicking on \emph{Permissions} shows the complete list of permissions for a
policy. Only super-admin users can make changes, but other admin users can
see the details of their own policy.

\subsection{Adding a Policy}

Choose a name that reflects your policy and submit it. You are now able to
edit the permissions for that policy. One of the first options is a priority
for use in the help directory (see~\ref{}). The lower the priority, the 
``closer'' this group is to the students, and therefore the
higher up such users will appear in the dynamic help directory.

You may now check and uncheck boxes as you wish to allocate permissions. 
In general, the recommendations are that
\begin{itemize}
  \item Very few, if any, users should be able to delete items - it damages
    the ability to use OPUS as an audit tool.
  \item Few users should be able to create items - these users should be well
    trained and able to act coherently as a group.
  \item Give day to day administrators more abilities to edit than create or delete.
  \item It is possible to create policies that can ``look but not touch'' for those
    users who only need read access to data.
\end{itemize}

In general a hierarchy of at least a couple of policies is recommended, to put
more experienced users in the higher bracket.

\section{Administrator Users}

Now you have created suitable policies, you are ready to create administrator
users and allocate them to policies. A policy provides a ceiling on the powers
of a user - it is possible to curtail them further as needed (see~\ref{}).

Clicking on the \opusacademicmenu{Directories -> Administrators} menu item will 
allow you to list and create new administrator users. As is often the case 
within OPUS, you should start by searching for the user to ensure they are not
already present before the \emph{add} button will present itself.

\subsection{Creating Administrators}

Adding an administrator can only be performed by a super-admin user, and quite
a bit of information is required. Add all the required fields, and note in
particular these others. When a user is created, they will be emailled their
initial username and password.

\subsubsection{Policy}

If no policy is defined, this user will not be able to do anything. Select an
appropriate policy having read the details above.

\subsubsection{Institutional Admin}

Sometimes a member of staff will need rights over the whole institution, but
not at a super-admin level. Rather than add them to each faculty, you can
choose to make them an institutional admin, in which case they will see all
programmes of study, and potentially all students. There should be 
\emph{very few} such users.

\subsubsection{Help Directory}

This dictates whether a user will appear in the dynamic help directory. You
should add members of staff who should be contacted by students and outside
companies here. Members of staff on holiday should be temporarily removed.

\subsubsection{Reg Number}

If the staff number is defined, it may be possible to login using a ``standard''
single sign-on method, depending on how your OPUS is configured. It is also
required for automatic saving of preferences.

\subsection{Editing Admin Details}

If you are a normal administrator you will only have permission to edit your own
account, and will not be able to edit other accounts.


\subsection{Promotion}
\index{root users}

You can (if you are a super-admin user), promote another admin to super-admin
status. This should be done with \emph{great care}!

\section{Super-Admin Users}

These users have a directory of their own, which allows them to be edited,
and demoted. Note you cannot directly create a super-admin user, you must 
create a normal admin user and promote them. This is not a bug.

You should generally keep \emph{very few} root accounts on your system.




\chapter{Initial Configuration}

After installation, it is usually necessary to copy the file \lstinline!config.php.dist! to
\lstinline!config.php! (in the \lstinline!include! directory) and customize its values.\footnote{Debian installation will take care of this aspect.}

\section{Organisation Details}

OPUS will need to know something about the way in which your organisation is
set-up. Clicking on this menu item will take you to an option to configure the
faculties that will use OPUS.

\subsection{Faculties}

Create a new faculty by using the \emph{add}
button. You will asked for a name, an optional web address, and an optional SRS
(Student Records System) code. If your system has a unique short identifier for
the faculty place it here.

Once a faculty is created, you can assign administrators that will have a role
over an entire faculty, or edit the schools within that faculty. Faculties can
be marked as \emph{archive} which means they are no longer active (do not
delete records unless you have added them in error, old records should be
archived).

\subsection{Schools}

This operates in an identical way to the faculty list, create schools and if
desired, assign administrators to them. You can edit and move a school to
another faculty once created by changing the faculty in the drop down box.

\subsection{Programmes}

Finally we come to programmes of study. Again, you should specify the name, and
code for the programme, and also decide on the \emph{CV Group} if you want
different handling for this course.

It is also possible, from the programme list, to assign administrators to an
individual programme, if no-one already assigned at institutional level,
faculty level or school level will do. Furthermore, clicking on assessment
allows you to configure the assessmentgroup that a programme belongs to. This
can change over the years, so it is possible to declare the start and end years
for each association. Again, you should not delete older associations.

\section{CV Groups}

This section allows you to configure the way in which CVs should be handled.
You can create as many CV Groups as you wish, and each one can contain as many
programmes as you wish from a single unusual case, to all your programmes of
study.

OPUS already ships with a Default Group which all programmes belong to unless
you change them.

\subsection{Adding a CV Group}

To add a CV Group simply click the \emph{add} button on this screen. You should
use a simple, easily understood name for the group, and you can add a detailed
description of the rationale for this group if you choose.

Editing the Permissions allows you to select if all PDSystem templates should 
be allowed (assuming the PDSystem is available to you), and whether custom CVs
should be allowed.

\subsection{Editing a CV Group}

Once you have created a group, when it appears in the listing, there will be
an option to specify how \emph{templates} are handled. If you have a coupled
PDSystem this will produce an up-to-date list of templates offered by the
PDSystem, and you can choose whether each template is allowed by that group
for applications for placement, and, whether any such template based CV must 
be approved before use.

You can also select a default template which will be suggested to the student
when they make an application.

Remember, if you have selected to allow All PDSystem templates, this screen
will have no effect.

\subsection{Assigned Programmes to Groups}

Once your CV Group is created, you should edit the programmes you wish to move
into it, and simply change the CV Group to your new choice.

\section{Assessment Groups}

Assessment Groups allow you to decide how students will be assessed within
OPUS. Before this will be of any use, you must have configured some assessments
within the Advanced menu.

A Default Scheme already exists within OPUS which has no assessments added to
it. You should be cautious about changing groups, and read this section
carefully before hand.

\subsection{Adding an Assessment Group}

Adding an assessment group is as simple as clicking on the \emph{add} option
within this section. All that can be added is a name (which as usual should be
clear and distinctive) and some comments to describe the rationale of the
group.

\subsection{Changing the Regime for a Group}

Once a group has been created, clicking on \emph{regime} within its listing
will allow you to alter the assessments associated with it.

\pdstip{Generally, you should \emph{never} alter the regime for any group that
is in current use unless you absolutely know what you are doing. You may make
it impossible to retrieve marks already recorded for this group.}

\subsubsection{Adding Assessments}

Clicking on \emph{add} allows you to choose from the available assessments to
add to your regime. You can use a given assessment more than once if that is
appropriate.

You will see a pull down box of all the available assessments, and you should
add a unique description, within your regime to describe this instance,
together with a weighting (number between 0 and 1) that describes how much this
is worth, for example 0.3 represents $30\%$. You should also choose who will
perform the assessment.

Next comes the year, which is the year, relative to placement that the
assessment will take place. That is, for an assessment that takes place in the
year of placement, enter zero, for an assessment that takes place the previous
year, enter $-1$ and $1$ for the year after placement and so on. The
\emph{start} and \emph{end} dates should be specified in the MMDD form when
the assessment should be due, and completed. This is used to guage when 
assessments are late.

\subsubsection{Editing and Removing Assessments}

You can edit and remove assessments, but again, this can cause already submitted
marks to become inaccessible (not deleted, but essentially lost in the database),
so only certain users can perform these actions, and generally they are for the
design phase of assessment groups not currently used.

\subsection{Changing Regime for a Programme}

Assessment needs change, but generally you should create a brand new assessment
group if that happens, and record that the programme changed from one to 
another in a given year. This is the only was to guarantee access to old
marks in the old regime as well as new ones.




\chapter{Courses, Groups \& Channels}

Before students are added to the system, it is import to configure schools and
courses. In addition, the concepts of groups and channels can help you deal
with courses in an efficient manner later, so it is worth giving some thought
to that now.

\section{Groups}

OPUS groups courses in two different ways: CV groups and Assessment groups.

A group provides a simple mechanism for dealing with a large number of courses
in one way. Groups can contain dozens of courses, or a single course that
requires special handling.

OPUS installs with a default group for CVs and a default group for Assessments;
if a course is not explicitly added to a group it is assumed to be in these
groups.

\subsection{CV Groups}
\label{CVGroups}

A CV group defines which CVs can be used by students on these courses.

\subsection{Assessment Groups}

An assessment group collects all the courses that are assessed in the same way.
You can define an assessment regime for each of these groups.

\section{Schools}
\index{schools}

A school is simply a division that contains a number of courses. You should 
create one or more schools. For each school it is possible to define its name
and any webpage it uses.

When editing a school record it is possible to label it as \opustext{archived}. Generally,
within OPUS data should not be deleted, since this destroys the audit capabilities
of the system. Labelling an old school as archived removes it from normal visibility
however.

You may also be able to define which administrators have permissions related to this
school. You can choose any administrator and a security policy. Normally you will
wish to use the default policy for the user, but it is also possible to give
curtailed access to a particular school by using a lower security policy.

\opustip{The default policy for a user always provides a ceiling on their abilities.
You cannot use a local policy to increase priviledges, only to reduce them.}

\section{Courses}

Editing courses is much like that of schools. When creating courses if only the
course code is specified OPUS attempts to retrieve the course details from your
SRS, otherwise you may supply full details.

\subsection{Administration Rights}

Just as for schools it is possible to define what administrators have powers over
this course.

\opustip{An administrator attached to the school this course is in already has
permissions for this course. Only use this for administrators who only have
access to one or two courses.}

\subsection{Course Directors}

You can also define a member of academic staff who is the course director. This
member of staff will obtain slighly augmented abilities (generally access to the
student directory (see~\ref{}) and most of its abilities for their students).

\subsection{CV Groups}

You can select a single CV group that this course belongs to. This will define
which CVs can be used by students on that course, and whether they require vetting.

\subsection{Assessment Groups}

A course is likely to belong to different assessment groups as the years progress and
assessment regimes change. In OPUS, this is handled by creating a new assessment group
and recording that students in subsequent years are members of the new group.

For each assessment group a course is allocated into a start and end year may be
specified. If for example the start year is 2006 and there is no end year then all
students seeking placement in 2006 or later belong to this group. Similarly an entry
showing no start year and an end year of 2005 indicates that students seeking
placement in 2005 and before belong to that group. Obviously it is possible to
specify both a start and end year.

\chapter{Importing Students}

Importing students is a vital step in populating OPUS with the data you need. There are several
approaches to this.

\section{Import from SRS}
\index{web services}

It is possible, with an appropriate web services layer for your student records system to import
students on mass. The form from \opusacademicmenu{Configuration -> Import Data} will allow
you to specify which year group you plan to import, and what academic year they will be seeking
placement. If it is during the 2007-2008 year for example, you should specify 2007 here. Generally
you should leave other information such as passwords blank, and normally you will wish to
import students with a status of ``Required'' indicating that they are seeking placement.

You will note a checkbox indicating that the import is a test. Try the import with the checkbox
ticked first of all and carefully ensure that everything went well, before removing the tick
to do the import for real.

You can repeatedly import the same students over and over again and OPUS will only import
any new students it has not seen before.

\section{From CSV files}

The University of Ulster uses CSV files to store student lists that can also be used for this
purpose, by browsing to the file first.

TODO: Specify the file format here.

\section{Individual Students}

It is also possible to add students individually using an option available from the Student
Directory.

\chapter{Student Directory}

The student directory is perhaps the most important administrative feature of OPUS. It provides
a mechanism to obtain or add information on individual students, and to extract information from
whole classes of students.

\section{Searching}

As for more of the directories for people within OPUS, there are two ways of
searching for students.

\subsection{Simple Searches}

To perform a simple search, simply click on the letter which begins the lastname
of the student you are looking for. OPUS will list all students that match, and
which you have permission to view. You can use the placement year shown to help
choose between entries that look like the one you need.

\subsection{Advanced Searches}

To perform an advanced search, you can use one of more of the search criteria
displayed.

\pdstip{The advanced search form will preserve your choices between usage if
you have a registration number entered for your user, and you logout correctly.}

\subsubsection{Search For}

The \emph{Search For} criteria can be used to match a fragment of a name, or a
reg number or username. Leave this field blank to match all students that meet
the other selected criteria.

\subsubsection{For placement in year}

This field, if specified, limits the search to students whose placement begins
in a specified year. For example, put in 2008 for students who will be on
placement in the 2008/2009 academic year. Leave this field blank to match all
students that meet the other selected criteria.

\subsubsection{From Programmes}

Here you will see, seperated by faculties and schools, a list of all the
programmes for which you will have permission to list students. If the list
is incorrect, you should contact one of the super-admin users to have them
correct it. Select which programmes you wish to search for students in. Note
that there are links to allow you to select or de-select programmes by a whole
school or faculty.

\subsubsection{Other Options}

At present there is only one \emph{other option} and it is used to augment the
output. The \emph{Show Timelines} switch allows you to produce a timeline
figure against each student, that shows you at a glance their pattern of
applications for vacancies. See below for more details.

\subsubsection{Sort Criterion}

The sort criterion allows you choose which field the results are to be sorted
by, for example, sorting by placement status is convenient for placing students
that still require placement at the top of the list.

\subsection{Student Lists}

Regardless of how you search for students, the matches are shown in the same
basic format.\footnote{Although of course, timelines will be present if you
requested them.}%

The first column is labelled \emph{Email}, and you select this tick box if you
wish to send a message to this student. There is a link to select or deselect
all the shown students at the top and bottom of the list. You can enter the
text of the email at the bottom, and optionally choose to be send a copy of
your email (which will be sent to come from your listed address).

Next you will see the student's \emph{Name} and \emph{Student Number}, the next
fields show when the student \emph{Last Accessed} OPUS, what year they expect
to be placed in, and their current placement status.

Finally you will see some options to manipulat the student, \emph{CV} is a
shortcut to the CV functions listed below, while \emph{Help} brings up the
help directory which is customized for the student at hand. The \emph{Edit}
option will bring up a detailed screen for viewing and manipulating that
student.

\section{Adding Students}

Adding students is usually done in bulk, via the \emph{Import Data} option,
but it is possible to add individual students, even if this is not recommended
since it is probably more error prone. You should first search for the student
to double check they are not already in OPUS. When the search results appear
the \emph{Add} button will be shown for adding a new student, simply complete
the form this is displayed.

\section{Editing a Student}

When you edit a student, the student's name will immediately appear on the 
top menu, to allow you to undertake actions with this student. This student
will remain on the menu, until you select another, or \emph{drop} him or her.

You should also see the student appear in the \emph{Recent} menu at the top,
allowing you to easily return to this student after editing others.

\subsection{Main View}

The main view for editing a student shows the most important information for
the student, such as the name, email address, registration number and placement
year. Also listed are the placement status, the programme of study, and the
Academic Tutor, if one is appointed.

At the top, you will see options to \emph{cancel} the edit, \emph{reset the
password}\footnote{This only assigns a new internal password, if you use
another system to authenticate students it could be of limited use}, and
\emph{manage the applications} made by the student.

\subsubsection{Placement History}

This section shows a summary of placement records for this student, each of
which can be edited. To add a new placement, click on \emph{manage applications}
and pick the application for which the student should be placed.

\subsubsection{Timeline}

This feature of OPUS allows a casual glance to determine the pattern of
applications for the student. A line is drawn that represents a certain
period of time, a thick black line indicated the student requires placement,
green indicates the student is placed, and grey represents any other 
status.

In the top right you will the total number of applications (made to open
vacancies within OPUS), and each red line represents a single application. An
arrow may appear to indicate that applications were made at some time outside
the timeline shown. The timeline should be accurate to within a few minutes,
but will not instantly register new applications.

\subsubsection{Photo}

A photograph of the student is shown if one exists. Clicking on the photo will
provide the fullsize image.

\subsubsection{Assessment}

The assessment grid for the student will be shown, if the assessment schedule
has been defined for the programme of study concerned. The grid shows, among
other things, the assessments that have been or will be taken, who will assess
them, their overall weighting, recorded marks and timeliness. Simply click on
view to view, or submit the assessment as appropriate. Assessments with marks
labelled by `--' are yet to be taken, otherwise a mark will be shown (even if
this is zero for marks with no weighting assigned).

\subsubsection{Configure Other Assessors}

Depending upon the assessment regime, it may be that other members of staff will
be assigned to mark other assessments. In that case, those assignments can be
made here.

\subsection{Home}

The \emph{Home} item on the student menu should bring you to a facsimile of the
student's home page, complete with the announcements that show on it. It may
not be an exact match in all regards.

\subsection{Vacancies}

The \emph{Vacancies} item allows you to search the vacancies in the normal way.
The student you have selected with stay attached to you, and when you
\emph{view} a vacancy you will be given an extra option to \emph{apply with
the selected student}.

\subsection{CVs}
\index{CVs}

This item provides information about the CVs that a student has completed, and
whether or not they can be used for applying for placement. The 
\emph{Description} shows the name of the CV and where it is stored, the 
\emph{Valid} column will show either that the CV is \emph{valid} and hence
allowable for application for placement, or the reason why it is not valid. A
final column will show if the CV has been \emph{approved} by a member of the
placement team. Whether or not approval is required as a prerequisite for a
valid CV depends on the configuration of the \emph{CV Group} for the programme
of student the student is enrolled in.

You can undertake a number of options with each CV, to \emph{view} them, 
\emph{approve} them or \emph{revoke} approval. As noted above, approval has
no effect unless the CV group requires it.

\subsection{Applications}

This item shows all the applications the student has filed on the system, 
together with their status and date and time of application. You can click on
\emph{place} to place the student with one of the applications. Otherwise you
can \emph{edit} or \emph{delete} existing applications, which should obviously
be done with care and an appropriate reason. Consider filing a note if this is
done.

\subsection{Channels}
\index{channels}

The channels which the student is automatically subscribed to are shown here.
Any errors should be handled by correcting the configuration of the channel
itself (see~\ref{}). If you wish to specifically add a student to another
channel, you can do so using the list below. Note that may not see all channels
depending upon your permissions. It is hugely preferable to have students
automatically added to channels so only add them directly as a last resort.

\subsection{Notes}

Here you can view, and file, \emph{notes} against a student. Notes are an
important feature of OPUS, and you should ensure you understand them before
proceeding to enter one since they \emph{cannot be edited or deleted}.

See more about notes elsewhere (\ref{}).

\subsection{Drop}

Finally you can \emph{drop} a student to remove them from your menu. This is
not necessary since a new student will always replace the existing one.

%
%
%

\chapter{Company Directory}

\section{Searching for companies}

\section{Adding and updating company records}

The company directory has an option to create a new company.

\opustip{It is vitally important you check if a company already exists in the database
before you create what might be a duplicate record.}

\section{Contacts}

\section{Resources}

\section{Companies and channels}

%
%
%

\chapter{Contact Directory}

\section{Searching for contacts}

\section{Managing contacts}

\section{Contacts and channels}

%
%
%

\chapter{Staff Directory}

\section{Searching for staff}

\section{Managing staff}

\section{Staff and students}

\section{Staff and channels}

%
%
%

\chapter{Help, Templates \& Resources}

\section{Help}

\subsection{Creating help}

\subsection{Editing help}

\section{Templates}

\subsection{Creating templates}

\subsection{Editing templates}

\section{Resources}

\subsection{Adding a resource}

\subsection{Resource information}

\subsection{Modifying a resource}

%
%
%

\chapter{Logs \& Status}

\section{Log files}

\subsection{Searching the log files}

\section{Status}

%
%
%

\chapter{Assessment}

\section{Assessments}

\subsection{Creating an assessment record}

\subsection{Creating an assessment skin}

\section{Creating an assessment regime}

%
%
%

\part{Academic Tutor Guide}

\chapter{Academic Tutors}

\part{Company HR Guide}

\chapter{Using OPUS to hire students}

\part{Workplace Supervisor Guide}

\chapter{Using OPUS to help supervise students}

\part{Student Guide}

%
%
%

\chapter{OPUS and PDP}

\section{PDP}

\section{The PDSystem}

%
%
%

\chapter{Preparing for and Getting a Placement}

\section{CVs}

\section{Covering Letters}

\section{e-Portfolios}

\section{Searching for vacancies}

\section{Applying for vacancies}

\section{Managing applications}

\section{Health \& Safety}

\section{Self assessments}

%
%
%


\chapter{Using OPUS while on placement}

\section{Announcements}

\section{Resources}

\section{Assessment}

%
%
%

\chapter{OPUS after your placement}

\section{Assessment}

\section{Learning the lessons}

%
% Index
%

\printindex
\markboth{Index}{}
\addcontentsline{toc}{chapter}{Index}

\end{document}



